\documentclass[12pt, letterpaper]{article}


%\bibliographystyle{imsart-nameyear}

%% Packages
\usepackage{amsmath,amsthm,amsfonts,amssymb,amscd}
\usepackage{calrsfs}
\usepackage{mathrsfs}
\usepackage{mathtools}
\usepackage{hyperref}
\usepackage[noabbrev]{cleveref}
\usepackage{float}
\usepackage{accents}
\usepackage{stmaryrd}
\usepackage{algpseudocodex}

\theoremstyle{plain}
\newtheorem{theorem}{Theorem}[section]
\newtheorem{proposition}{Proposition}[section]
\newtheorem{lemma}{Lemma}[section]
\newtheorem{claim}{Claim}[section]
\newtheorem{corollary}{Corollary}[section]

\theoremstyle{remark}
\newtheorem{remark}{Remark}
\newtheorem{assumption}{Assumption}

\def\d{\mathrm{d}}


\newcommand{\st}{\,:\,}
%\newcommand{\norm}[1]{
%  \relax\if@display
%      % 0 = \displaystyle
%      \left\| #1\right\|
%  \else
%      % all other styles
%      \| #1 \|
%  \fi}
\newcommand{\norm}[1]{\| #1 \|}
\crefname{assumption}{assumption}{assumptions}

\begin{document}


\title{Title}
\author{Author(s)}

\maketitle


%\begin{abstract}
%DRP Winter 2024
%\end{abstract}

\section{Section}

This is the report for DRP 2024 at McGill University and as to do with SGD~\cite{wiki:sgd}.

\subsection{Subsection}

This is a subsection. In \LaTeX, you can use inline math like this: \(\int_{0}^{x}f(x)\d x\). You can also put longer/bigger equations in math mode like this:
\[
    \begin{bmatrix}
        A & B^{T} \\
        B & Q 
    \end{bmatrix}^{-1}
    = \begin{bmatrix}
        (A-B^{T}Q^{-1}B)^{-1} & -(A-B^{T}Q^{-1}B)^{-1}B^{T}Q^{-1} \\
        -Q^{-1}B(A-B^{T}Q^{-1}B)^{-1} & Q^{-1} + Q^{-1}B(A-B^{T}Q^{-1}B)^{-1}B^{T}Q^{-1}
    \end{bmatrix}.
\]
Use punctuation like you would do in a regular sentence. 

\subsubsection{Subsubsection}
If you want to refer to an equation, use
\begin{equation}\label{eq:pythagorean}
    a^{2} + b^{2} = c^{2}.
\end{equation}

Later in the text, you can refer to \cref{eq:pythagorean} using the cleveref package. You can do the same for theorem, lemma, etc.

\begin{theorem}\label{theorem:first_theorem}
    First theorem
\end{theorem}

\begin{theorem}[Named theorem]\label{theorem:named_theorem}
    Named theorem
\end{theorem}

\begin{lemma}
    Some lemma
\end{lemma}

\begin{proof}
    Put the proof here
\end{proof}

\begin{remark}
    You can put \emph{emphasis} on a word using the command \\emph.
\end{remark}

\begin{algorithmic}[line numbering]
    \State Initialize some things
    \Statex Continuing
    \For{\(n=1,\ldots, 10\)}
        \If{True}
        \State \Call{Some function}{x,y,z}
        \Else
        \State Do nothing
        \EndIf
    \EndFor
    \State \Output Done
\end{algorithmic}

\bibliographystyle{alpha}
\bibliography{biblio}

\end{document}

